% !TEX root = ../thesis.tex
\chapter{Formalizing the Problem} \section{Introduction} The existing research
on school choice and course allocation would seem to suggest that for our
problem, which only features one-sided preferences as opposed to preferences on
both sides of the market, we follow a fair division approach similar to that of
Budish’s one-sided matching. However, given the practical application of our
problem as a mechanism for assigning every student to one specific class, we
determined that constructing a notion of stability makes more sense than
evaluating envy in the matching. Our solution is driven by the goal of assigning
HUM classes that are balanced both in size and student interests, so we want to
minimize the ability of students to swap with each other post-assignment as much
as possible so as not to disrupt the balance. If no student wishes to swap with
another student, then we can consider the assignment stable. This is the
framework with which we shall formalize our model.

\section{The Model}

The basic components of our model are the two sides of our market: a set of
students and a set of HUM class sections, each of which have a certain number of
seats. Though we are working in a bipartite setting, only one side of our market
has preference over the other (i.e. students prefer certain classes, classes are
inanimate and therefore indifferent to students as far as we know). Students
also do not necessarily have ranked preference over every single available
section; they likely equally do not want to take any sections that conflict with
their other selected courses. So, preference lists will be incomplete, with
those that students cannot take not being included at all. We will denote the
preference list for student $s \in S$ as $P_s$. \newpage

\begin{definition} The \emph{HUM-assignment} problem consists of a finite set of
  students, $S$, and a finite set of class sections $C$. Each class section $c
  \in C$ has a capacity $q_c$ representing the number of available seats in the
  section. Each student $s \in S$ has a strict preference relation $\succ_s$
over $C \cup \{\emptyset\}$, where $\emptyset$ represents the outside option of
a student such that $\emptyset \succ_s c$ for some $c \in C$ denotes that the
class is not an option for the student. \end{definition} \begin{definition} A
\emph{1-sided matching (1M)} of students to class sections is a function $\mu :
S \rightarrow C \cup \emptyset $ such that \begin{itemize} \item $\mu(s) =
  \emptyset \text{ or } \mu(s) \succ_s \emptyset \text{ for all }s \in S$ \item
$|\{s \in S : \mu(s) = c\}| \leq q_c$ \text{ for all } $c \in C$ \end{itemize}
\end{definition}

Finding a 1-sided matching would solve an instance of the HUM-assignment problem
as defined above. However, we still have to introduce the last aspect of the
problem: the \emph{proportionality} constraint.

Students can be divided into three types based on the way Reed College divides
majors into groups: Group 1 for language, arts, and literature, Group 2 for
social sciences, Group 3 for math and lab sciences. To implement the HUM
philosophy, which is that the class provides an opportunity for students of all
knowledge backgrounds and academic interests to discuss relevant texts, we want
to ensure a balanced spread of each type of student between sections. This means
avoiding scenarios such as all chem 101 students with the same lab being placed
in the same section due to the overwhelming similarity in their schedules. This
will come in the form of quotas or caps on the number of seats per section that
can be assigned to each type of student, so each class will be divided evenly
into four types of available seats: A, B, C, and free seats to allow for some
flexibility.

\begin{definition} Under the proportionality constraint, $S$ is partitioned into
three sets: $A, B,edge backg S$, where $A \cup B \cup C = S$ and $A \cap B \cap
C = \emptyset$. Let students $s_1, s_2 \in A$ be denoted $a_1, a_2$.
\end{definition}

\begin{definition} For each class $c$ with capacity $q_c$, let $q_c^t = \lfloor
q_c / 4 \rfloor$ for each $t \in \{A, B, C, F\}$, where $A, B, C$ correlate to
the types of students and $F$ represents the set of free seats. \end{definition}

\begin{definition} A \emph{proportional 1-sided matching (P1M)} builds on our
definition of 1-sided matching by adding the following quality: $|\{s \in A :
\mu(s) = c\}| \leq q_c^A + q_c^F$. \end{definition}

To make things clearer, we can represent our problem as a graph to provide us
with a visualization of the problem (see Figure \ref{fig:model}).

\begin{figure}[H] \centering
\includegraphics[width=0.85\linewidth]{figures/model.jpeg}\caption{Graph
illustrating a proportional 1-sided matching} \label{fig:model} \end{figure}

In a graph representing the HUM-assignment problem, the vertices are comprised
of students and classes, and an edge is drawn between a student and a class if
the class has an available seat for a student of the same type and the class is
on the student's preference list. We can formally define this representation.  

\begin{definition} Let $G = (V, E)$ be a bipartite graph where: \begin{itemize}
\item $V = S \cup C$ \item $E = \{(s, c) \mid s \in S, c \in C, c \succ_s
\emptyset\}$ \end{itemize} \end{definition}

In this form, the graph only represents our problem, not a solution. For a graph
to represent a solution or a matching, each student could only have one edge
connecting them to a single class in a way that represents class capacities. So,
we need a way to narrow down which edges lead to a viable solution. That means
we need a way to judge edges against each other; we need to assign value to
them. To accomplish this, we can give each edge a weight that represents how
conducive it is to the constraints of our problem. Recall that we want to find a
solution that satisfies two goals: every student is assigned to a class they
prefer, and all reserved seats are filled by students of that reserved type. We
can assign weights to the edges based on these goals such that a higher weight
indicates that edge embodies more of the qualities we want in our solution.
Because we have two goals that are equally important, we can simply increment
the weight of each edge by 1 for each goal it satisfies. 

[DEFINE WEIGHTS FORMALLY]

[INSERT IMAGE HERE OF THE WEIGHTED GRAPH]

Now, we have a way to compare edges to each other to allow us to find a
solution. Based on the way we've set up our weights, we know that the best
available solution will be the graph with the maximum possible weight when only
one edge remains for each student. Our max-weight matching then has the
properties required by our P1M definition, being that students of each type are
divided relatively evenly to classes they prefer to be in, as well as maximizing
the overall satisfaction of students with their assignments. 

[DEFINE MAX-WEIGHT MATCHING]

A max-weight matching would not only serve as a P1M solution, it would also by
nature of the way we assign weights ensure the solution is one in which the most
students are happy with their match. So, viewing the problem through the lens of
a weighted graph allows us to both solve the problem, and ensure the solution is
the most effective.


\section{The Challenge of Stability} There is one more aspect to the problem we
must consider. KEEP WRITING Notably, “stability” in our case adopts a different
meaning than in a traditional stable matching problem. Because we are working
with one-sided preferences, we do not consider “cheating” as a violation of
stability between students and classes, but rather between students themselves.
So, stability can more formally be defined by saying there exists no blocking
pair between student $a$ and student $b$ such that $\mu(b) >_a \mu(a)$ and
$\mu(a) >_b \mu(b)$.

\begin{theorem} There is not always a stable matching with quotas (caps) and
incomplete preferences. \end{theorem}

\begin{proof} Let $C = \{ c_1, c_2\}$, where there are two types of student
  $\alpha$ and $\beta$ such that each $c \in C$ can have at most one student of
  each type. Let $S = \{ A1, B1, A2, B2 \}$ where students $A1, A2$ are of type
  $\alpha$ and $B1, B2$ are of type $\beta$. Let the preference lists be as
  follows: \[ \begin{array}{ll} A1: & c_1           \\ B1: & c_2 \\ A2: & c_1
  \succ c_2 \\ B2:      & c_2 \succ c_1 \end{array} \]

  In this case, regardless of the order of matching or algorithm, there is no
  stable matching. The $A$ students will both wish to be matched to $c_1$, which
  only has one $\alpha$ spot, and both $B$ students wish to be matched to $c_2$
  which only has one $\beta$ spot. So, when the first round of matching is done,
that means that there is still an open $\beta$ spot for $c_1$ and an open
$\alpha$ spot for $c_2$, which the unhappy $A$ and $B$ students can then switch
into on their own, respectively. \end{proof}

Defining stability in this way is, then, too strict. Recall that even
determining whether there is a stable matching or not is NP-complete in many
scenarios similar to this \cite{santhini}. So, we need to relax our constraints
to allow for some notion of stability. For now, let’s simplify the problem so we
can build back up to a working definition for stability. At a basic level,
before we even think about stability, we want to work in an environment where we
can be sure an algorithm would respect student preferences, such that no student
is assigned to a class with which they have a time conflict.

We can start by looking at an environment where we have 0/1 preferences and
therefore no notion of stability. Let $C$ denote the set of seats over all
classes, and $S$ the set of students. We will put lower bounds on the seats,
such that the number of seats in each class will be reserved evenly among each
type of student, with an extra group of seats that is free to any student. We
will let $C_r$ be the set of all reserved seats. In this case, we have an
environment well-suited to the weighted graph matching problem. We can say  $G =
(V, E)$ such that $V = S \cup C$ and each edge goes from some $s$ to some $c$ if
and only if $c \in P_s$, where $P_s$ is the preference list of $s$. Each edge
$e$ has some weight $w(e)$. Let the weight be as follows: \[ w(e) =
\mathbf{1}\{\text{if seat $c$ is reserved for $s$'s type}\} +
\mathbf{1}\{\text{if student $s$ prefers class $c$}\}. \]

\begin{theorem} \label{max-weight} If a matching that respects student
preferences and seat reservations exists, it will be the max-weight matching
where $w(\mu) = |S| + |C_r|$. \end{theorem}

\begin{proof} Let $\mu$ be a matching that assigns every student to a seat they
  prefer while also filling reserved seats first, such that all reserved seats
  are filled before the overflow seats. By construction of the weights, we can
  determine a lower and upper bound for $w(\mu)$.

  \begin{enumerate} \item \textbf{Lower bound of $\mu$ }  Each edge exists only
    if the student prefers that seat, so every edge in $\mu$ contributes at
    least 1. By our assumption of characteristics of $\mu$, every student in
    $\mu$ got a seat they prefer, so summing over all students contributes at
    least $|S|$ to $w(\mu)$. We also assumed each reserved seat was assigned to
    a student in $\mu$, which adds an extra +1 to the weight for the edge from
    each of those reserved seats. Summing over all reserved seats gives an
    additional contribution of $|C_r|$, because the rest of the seats are still
    1-weight edges. Therefore, we can bound $\mu$: \[ w(\mu) \geq |S| + |C_r|.
    \]

    \item \textbf{Upper bound of $\mu$} From our definition, we know each edge
      weight is at most 2. Similarly to the lower bound, we know $\mu$ has $|S|$
      edges because each student was matched to one of their preferences.
      However, because only the reserved seats contribute an additional weight
      to an edge, we know that the number of 2-weight edges $\leq |C_r|$. So,
      the maximum summed weight of all edges must be $2|C_r| + |S| - |C_r|$,
      representing the maximum number of 2-weight edges plus all students
    assigned to an overflow seat. Note that is equivalent to $|S| + |C_r|$.
  Therefore, we can bound $\mu$: \[ w(\mu) \leq |S| + |C_r|. \] \end{enumerate}

So, $\mu$ is a maximum-weight graph with a lower bound of $|S| + |C_r|$ and an
upper bound of $|S| + |C_r|$, so it must be that a matching that fully respects
student preferences and seat reservations has a weight $w(\mu) = |S| + |C_r|$.
\end{proof}

Theorem \ref{max-weight} shows us that using the max-weight algorithm, we can
assign each student to a class they prefer while respecting diversity if such a
matching exists. From here, we can begin to reintroduce stability. While it may
be too computationally difficult to find stability between every student,
perhaps we can start by comparing just students of the same type.  We are
introducing a new notion of stability here, where we will consider a matching
stable for our purposes if it exhibits \textit{intra-group stability}. This
means that for any type of student, that student does not form a blocking pair
with another student of the same type.

Formally, a matching $\mu$ is stable if, for any two students $a_1$ and $a_2$ of
type $A \subset S$, \\ $\mu(a_1) >_{a_1} \mu(a_2)$  or $\mu(a_2) >_{a_2}
\mu(a_1)$ such that there exist no intra-group blocking pairs.

For now, inter-group instability, or blocking pairs involving students of
different types, does not interfere with our notion of stability. Though we
would ideally like to guarantee stability between all students, looking at the
problem through the lens of intra-group stability allows us to give some
theoretical guarantee that helps minimize the amount of swapping that will occur
in an assignment while remaining computationally feasible, as opposed to turning
to less-reliable heuristic solutions.

