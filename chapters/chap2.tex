% !TEX root = ../thesis.tex
\chapter{Formalizing the Problem} \section{Introduction} The existing research
on school choice and course allocation would seem to suggest that for our
problem, which only features one-sided preferences as opposed to preferences on
both sides of the market, we follow a fair division approach similar to that of
Budish’s one-sided matching \cite{budish}. However, given the practical application of our
problem as a mechanism for assigning every student to one specific class, we
determined that constructing a notion of stability makes more sense than
evaluating envy in the matching. Our solution is driven by the goal of assigning
HUM classes that are balanced both in size and student interests, so we want to
minimize the desire of students to swap with each other post-assignment as much
as possible so as not to disrupt the balance. If no student wishes to swap with
another student, then we can consider the assignment stable. This is the
framework with which we shall formalize our model.

\section{The Model}

The basic components of our model are the two sides of our market: a set of
students and a set of HUM class sections, each of which have a certain number of
seats. Though we are working in a bipartite setting, only one side of our market
has preference over the other (i.e. students prefer certain classes, classes are
inanimate and therefore indifferent to students...as far as we know). Students
also do not necessarily have ranked preference over every single available
section; they likely equally do not want to take any sections that conflict with
their other selected courses. So, preference lists will be incomplete, with
those that students have a conflict with not being included in their ranking.
\newpage

\begin{definition}
  The \emph{HUM-assignment} problem consists of a finite set of
  students, $S$, and a finite set of class sections $C$. Each class section $c
    \in C$ has a capacity $q_c$ representing the number of available seats in the
  section. Each student $s \in S$ has a strict preference relation $\succ_s$
  over $C \cup \{\emptyset\}$, where $\emptyset$ represents the outside option of
  a student such that $\emptyset \succ_s c$ for some $c \in C$ denotes that the
  class is not an option for the student.
\end{definition}
\begin{definition}
  A
  \emph{1-sided matching (1M)} of students to class sections is a function $\mu :
    S \rightarrow C \cup \emptyset $ such that
  \begin{itemize}
    \item
          Each student is assigned to either a class they prefer or no
          class: \\ $\mu(s) =
            \emptyset \text{ or } \mu(s) \succ_s \emptyset \text{ for all }s \in
            S$
    \item
          No class exceeds its capacity:
          $|\{s \in S : \mu(s) = c\}| \leq q_c$ \text{ for all } $c \in C$
  \end{itemize}
\end{definition}

Finding a 1-sided matching would solve an instance of the HUM-assignment problem
as defined above. However, we still have to introduce the last constraint of the
problem: \emph{proportionality}.

Students can be divided into three groups based on the way Reed College divides
majors: Group 1 for language, arts, and literature, Group 2 for
social sciences, Group 3 for math and lab sciences. To implement the HUM
philosophy, which is that the class provides an opportunity for students of all
knowledge backgrounds and academic interests to discuss relevant texts, we want
to ensure a balanced spread of each group of student between sections. This means
avoiding scenarios such as all chem 101 students with the same lab being placed
in the same section due to the overwhelming similarity in their schedules. This
will come in the form of quotas or caps on the number of seats per section that
can be assigned to each group of student, so each class will be divided 
into four groups of available seats: A, B, C, and free seats to allow for some
flexibility.

\begin{definition}
  Under the proportionality constraint, $S$ is partitioned into
  three sets: $A, B, C \subset S$, where $A \cup B \cup C = S$ and $A \cap B \cap
    C = \emptyset$. Let students $s_1, s_2 \in A$ be denoted $a_1, a_2$.
\end{definition}

\begin{definition}
  For each class $c$ with capacity $q_c$, let $q_c^t = \lfloor
    q_c / 4 \rfloor$ for each $t \in \{A, B, C, F\}$, where $A, B, C$ correlate to
  the groups of students and $F$ represents the set of free seats.
\end{definition}

\begin{definition}
  A \emph{proportional 1-sided matching (P1M)} builds on our
  definition of 1-sided matching by adding the quality that no class exceeds its capacity for each
  group of student:
  $|\{s \in A :
    \mu(s) = c\}| \leq q_c^A + q_c^F$.
\end{definition}

To make things clearer, we can represent our problem as a graph to provide us
with a visualization of the problem (see Figure \ref{fig:model}).

\begin{figure}[H] \centering
  \includegraphics[width=0.85\linewidth]{figures/model.jpeg}\caption{Graph
    illustrating a proportional 1-sided matching} \label{fig:model}
\end{figure}

In a graph representing the HUM-assignment problem, the vertices are comprised
of students and classes, and an edge is drawn between a student and a class if
the class is
on the student's preference list. We can formally define this representation.

\begin{definition}
  Let $G = (V, E)$ be a bipartite graph where:
  \begin{itemize}
    \item
          $V = S \cup C$
    \item
          $E = \{(s, c) \mid s \in S, c \in C, c \succ_s
            \emptyset\}$
  \end{itemize}
\end{definition}

In this form, the graph only represents our problem, not a solution. For a graph
to represent a solution or a matching, each student would only have one edge
connecting them to a single class in a way that adheres to class capacities. So,
we need a way to narrow down which edges lead to a viable solution. That means
we need a way to judge edges against each other: we need to assign value to
them. To accomplish this, we can give each edge a weight that represents how
conducive it is to the constraints of our problem. Recall that we want to find a
solution that satisfies two goals: every student is assigned to a class they
prefer, and all reserved seats are filled by students of that reserved group. We
can assign weights to the edges based on these goals such that a higher weight
indicates that edge embodies more of the qualities we want in our solution.
Because we have two goals that are equally important, we can simply increment
the weight of each edge by 1 for each goal it satisfies. We can now supplement
our previous definition of a graph.\newpage

\begin{definition}
  \label{weighted-def}
  Let $G = (V, E)$ be a bipartite graph where:
  \begin{itemize}
    \item
          $V = S \cup C$
    \item
          $E = \{(s, c) \mid s \in S, c \in C, c \succ_s
            \emptyset\}$
    \item
          $w : E \rightarrow \{1 , 2 \} $ is a weight function
          defined as \[ w(s, c) = \mathbf{1}\{\text{if student $s$ prefers class
              $c$}\} + \mathbf{1}\{\text{$c$ is reserved for $s$'s
              group}\} \]
  \end{itemize}
\end{definition}

\begin{figure}[H] \centering
  \includegraphics[width=0.7\linewidth]{figures/weighted-example.jpeg}\caption{Graph
    illustrating weighted edges for one student} \label{fig:weighted-example}
\end{figure}

Now, we have a way to compare edges to each other to allow us to find a
solution if one exists. Based on the way we've set up our weights, we know that the best
available solution will be the graph with the maximum possible weight when
only one edge remains for each student. Our max-weight matching then has the
properties required by our P1M definition, being that students of each group
are divided relatively evenly to classes they prefer to be in, as well as
maximizing the overall satisfaction of students with their assignments.

\begin{definition}
  A matching $\mu$ is a \emph{maximum-weight matching (MWM)}
  if for all P1M matchings $\mu'$: \[ \sum_{s \in S} w(s, \mu(s)) \geq \sum_{s
      \in S} w(s, \mu'(s)) \]
\end{definition}

Observe that a maximum-weight matching would not only serve as a P1M
solution, but it would also (by nature of the way we assign weights based on
student preferences) ensure the solution is one in which the greatest number
of students are happy with their match. So, viewing the problem through the
lens of a weighted graph allows us to both solve the problem, and provide
the solution that gives the most student satisfaction.

\section{The Challenge of Stability} We've determined how to find a solution
to the problem we defined, and even find a solution that maximizes student
happiness, but there is one more aspect to the problem we must consider.
Recall that matching is often studied under the analysis of
\textbf{stability}. Notably, “stability” in our case adopts a different
meaning than in a traditional stable matching problem. Because we are
working with one-sided preferences, we do not consider “cheating” as a
violation of stability between students and classes, but rather between
students themselves. That is, we can consider a matching stable if there are
no blocking pairs \emph{between students}.

\begin{definition}
  \label{def:stability}
  A matching $\mu$ is \emph{stable} if there exists no blocking pair between
  students
  $s_1, s_2 \in S$ such that $\mu(s_2) >_{s_1} \mu(s_1)$ and $\mu(s_1) >_{s_2}
    \mu(s_2)$.
\end{definition}

This notion of stability proves difficult for our problem, however. If we look
back at our definition of the HUM-assignment problem with class capacities and
incomplete preferences, we will find that in some instances it is possible there
may not even exist a stable matching to find.

\begin{theorem}
  There is not always a stable matching with quotas and
  incomplete preferences.
\end{theorem}

\begin{proof}
  Let $C = \{ c_1, c_2\}$, where there are two groups of student
  $A$ and $B$ such that for each $c \in C, q_c^A, q_c^B = 1$ (each class can
  have at most one student of
  each group). Let $S = \{ a_1, b_1, a_2, b_2 \}$. Let the preference lists be as
  follows: \[
    \begin{array}{ll} a_1: & c_1           \\ b_1: & c_2 \\ a_2: & c_1 \succ c_2 \\
             b_2:      & c_2 \succ c_1
    \end{array}
  \]

  In this case, regardless of the order of matching or algorithm, there is no
  stable matching. The $A$ students will both wish to be matched to $c_1$, which
  only has one $A$ spot, and both $B$ students wish to be matched to $c_2$
  which only has one $B$ spot. So, when the first round of matching is done,
  that means that there is still an open $B$ spot for $c_1$ and an open
  $A$ spot for $c_2$, which the unhappy $A$ and $B$ students can then switch
  into on their own, respectively.
\end{proof}

Defining stability in this way is, then, too strict. Recall that even
determining whether there is a stable matching or not is NP-complete in many
scenarios similar to this one \cite{santhini}. So, we need to relax our constraints
to allow us to explore some other notion of stability. For now, let’s simplify the problem. At a basic level,
before we even think about stability, we want to work in an environment where we
can be sure an algorithm would respect student preferences, such that no student
is assigned to a class they did not list as an option. \newpage

To temporarily remove our notion of stability, we can alter our preferences from a
ranked system to what is known as \textbf{0/1 preferences}. In 0/1 preferences,
students assign a class
a 1 if they are okay with it and a 0 if it is not an option for them. This way,
there cannot be a notion of stability because all preferred classes have the
same rank. Now, for ease of notation, we
will let $C_r$ be the set of reserved seats across all classes. Observe that
this includes all seats other than the set of free seats, or formally: 
 \[
    |C_r| = \sum_{\substack{c \in C \\ t \neq F}} q_c^t
  \] Notably, this
simplified environment still lends itself well to the same weighted graph construction
from definition \ref{weighted-def}, so we can continue to use it to represent our
problem.

\begin{theorem}
  \label{max-weight} A matching fills all reserved seats while only assigning students to classes they prefer if and only if it is the maximum-weight matching
  where $w(\mu) = |S| + |C_r|$.
\end{theorem}

\begin{proof}
  First we will prove the forward direction. Let $\mu$ be a matching that
  assigns every student to a seat they
  prefer while also filling reserved seats first, such that all reserved seats
  are filled before free seats. By construction of the weights, we can
  determine a lower and upper bound for $w(\mu)$. 


  \begin{enumerate}
    \item
          \textbf{Lower bound of $\mu$:} Each edge exists only
          if the student prefers that seat, so every edge in $\mu$ contributes at
          least 1. By our assumption of characteristics of $\mu$, every student in
          $\mu$ got a seat they prefer, so summing over all students contributes at
          least $|S|$ to $w(\mu)$. We also assumed each reserved seat was assigned to
          a student in $\mu$, which adds an extra +1 to the weight for the edge from
          each of those reserved seats. Summing over all reserved seats gives an
          additional contribution of $|C_r|$, because the rest of the seats are still
          1-weight edges. Therefore, we can bound $\mu$: \[ w(\mu) \geq |S| + |C_r|.
          \]

    \item
          \textbf{Upper bound of $\mu$:} From our definition, we know each edge
          weight is at most 2. Similarly to the lower bound, we know $\mu$ has $|S|$
          edges because each student was matched to one of their preferences.
          However, because only the reserved seats contribute an additional weight
          to an edge, we know that the number of 2-weight edges $\leq |C_r|$. So,
          the maximum summed weight of all edges must be $2|C_r| + |S| - |C_r|$,
          representing the maximum number of 2-weight edges plus all students
          assigned to an overflow seat. Note that is equivalent to $|S| + |C_r|$.
          Therefore, we can bound $\mu$: \[ w(\mu) \leq |S| + |C_r|. \]
  \end{enumerate}

  So, $\mu$ is a maximum-weight graph with a lower bound of $|S| + |C_r|$ and an
  upper bound of $|S| + |C_r|$, so it must be that a matching that fully respects
  student preferences and seat reservations has a weight $w(\mu) = |S| + |C_r|$.

  Now, we prove the backwards direction. Let $\mu$ be a maximum-weight matching
where $w(\mu) = |S| + |C_r|$. The term $|S|$ implies each student was assigned
to a class. By our construction of $G$ such that an edge between a student and a
class only exists if the student prefers that class, the fact that each student
is assigned to a class must mean it is on their preference list. Additionally,
the $|C_r|$ term implies every reserved seat is assigned, because by our weight
function each edge only has additional weight $+1$ if the edge is between a
student of the same group as the seat. So, a maximum-weight matching where 
$w(\mu) = |S| + |C_r|$ is a matching that fills all reserved seats while only
assigning students to classes they prefer.
\end{proof}

Theorem \ref{max-weight} shows us that we can identify a way to 
assign each student to a class they prefer while respecting proportionality
constraints if such a
matching exists. From here, we can begin to reintroduce stability. While it may
not be feasible to find stability between every student,
perhaps we can start by comparing just students of the same group.  We are
introducing a new notion of stability here, where we will consider a matching
stable if it exhibits \textbf{intra-group stability}. This
means that for any group of student, that student does not form a blocking pair
with another student \textbf{of the same group}.

\begin{definition}
  A matching $\mu$ is \emph{intra-group stable} if, for any two students $t_1$
  and $t_2$ for all
  groups $T \subset S$, $\mu(t_1) >_{t_1} \mu(t_2)$  or $\mu(t_2) >_{t_2}
    \mu(t_1)$ such that there exist no intra-group blocking pairs.
\end{definition}

With our new notion of intra-group stability, we can now add to our
understanding of what makes a viable solution to our HUM-assignment problem.

\begin{definition}
  An \emph{intra-group stable proportional 1-sided matching (ISP1M)} is a function $\mu :
    S \rightarrow C$ such that
  \begin{itemize}
    \item
          Each student is assigned to a class they prefer: \\
          $\mu(s) \succ_s \emptyset \text{ for all }s \in
            S$
    \item
          No class exceeds its capacity:
          $|\{s \in S : \mu(s) = c\}| \leq q_c$ \text{ for all } $c \in C$

    \item
          No class exceeds its capacity for each
          group of student: \\ $|\{s \in T : \mu(s) = c\}| \leq q_c^T + q_c^F$
    \item
          There exist no intra-group blocking pairs: \\ for all $t_1, t_2 \in T \subset S$, $\mu(t_1) >_{t_1} \mu(t_2)$ or
          $\mu(t_2) >_{t_2} \mu(t_1)$
  \end{itemize}
\end{definition}

For now, our goal will be to develop an algorithm that finds an ISP1M to solve
our problem. We will classify our previous definition of stability (definition
\ref{def:stability}) as
\textbf{inter-group stability} (blocking pairs involving students of different
groups) to draw a distinction between the two notions.
For now, inter-group instability does not interfere with our search for an intra-group stable
solution. Though we
would ideally guarantee stability between all students, looking at the
problem through the lens of intra-group stability allows us to give some
theoretical guarantee about minimizing the amount of swapping that will occur
in an assignment while remaining computationally feasible, as opposed to turning
to less-reliable heuristic solutions to try to solve the broader conception of the problem.

